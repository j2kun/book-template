
\frontmatter

\chapter{Our Goal}

This book has a straightforward goal: to teach you how to engage with
mathematics.

Let's unpack this. By ``mathematics,'' I mean the universe of books, papers,
talks, and blog posts that contain the meat of mathematics: formal definitions,
theorems, proofs, conjectures, and algorithms. By ``engage'' I mean that for
any mathematical topic, you have the cognitive tools to actively progress
toward understanding that topic. I will ``teach'' you by introducing you
to---or having you revisit---a broad foundation of topics and techniques that
support the rest of mathematics. I say ``with'' because mathematics requires
active participation.

We will define and study many basic objects of mathematics, such as
polynomials, graphs, and matrices. More importantly, I'll explain \emph{how to
think} about those objects as seasoned mathematicians do. We will examine the
hierarchies of mathematical abstraction, along with many of the softer skills
and insights that constitute ``mathematical intuition.'' Along the way we'll
hear the voices of mathematicians---both famous historical figures and my
friends and colleagues---to paint a picture of mathematics as both a messy
amalgam of competing ideas and preferences, and a story with delightfully
surprising twists and connections.  In the end, I will show you how
mathematicians think about mathematics.

So why would someone like you\footnote{Hopefully you're a programmer;
otherwise, the title of this book must have surely caused a panic attack.} want
to engage with mathematics? Many software engineers, especially the sort who
like to push the limits of what can be done with programs, eventually come to
realize a deep truth: mathematics unlocks a \emph{lot} of cool new programs.
These are truly novel programs. They would simply be impossible to write (if
not inconceivable!) without mathematics. That includes programs in this book
about cryptography, data science, and art, but also to many revolutionary
technologies in industry, such as signal processing, compression, ranking,
optimization, and artificial intelligence. As importantly, a wealth of
opportunity makes programming more fun! To quote Randall Munroe in his XKCD
comic \emph{Forgot Algebra,} ``The only things you HAVE to know are how to make
enough of a living to stay alive and how to get your taxes done. All the fun
parts of life are optional.'' If you want your career to grow beyond shuffling
data around to meet arbitrary business goals, you should learn the tools that
enable you to write programs that captivate and delight you. Mathematics is one
of those tools.

Programmers are in a privileged position to engage with mathematics.  As a
programmer, you eat paradigms for breakfast and reshape them into new ones for
lunch. Your comfort with functions, logic, and protocols gives you an intuitive
familiarity with basic topics such as boolean algebra, recursion, and
abstraction. You can rely on this to make mathematics less foreign, progressing
all the faster to more nuanced and stimulating topics. Contrast this to most
educational math content aimed at students with no background and focusing on
rote exercises and passing tests.  As a bonus, programming allows me to provide
immediate applications that ground the abstract ideas in code. In each chapter
of this book, we'll fashion our mathematical designs into a program you
couldn't have written before, to dazzling effect. The code is available on
Github,\footnote{\webpageurl} with a directory for each chapter.

All told, this book is \emph{not} a textbook. I won't drill you with exercises,
though drills have their place. We won't build up any particular field of
mathematics from scratch. Though we'll visit calculus, linear algebra, and many
other topics, this book is far too short to cover everything a mathematician
ought to know about these topics. Moreover, while much of the book is
appropriately rigorous, I will occasionally and judiciously loosen rigor when
it facilitates a better understanding and relieves tedium. I will note when
this occurs, and we'll discuss the role of rigor in mathematics more broadly.

Indeed, rather than read an encyclopedic reference, you want to become
\emph{comfortable} with the process of learning mathematics. In part that means
becoming comfortable with discomfort, with the struggle of understanding a new
concept, and the techniques that mathematicians use to remain productive and
sane. Many people find calculus difficult, or squeaked by a linear algebra
course without grokking it. After this book you should have a core nugget of
understanding of these subjects, along with the cognitive tools that will
enable you dive as deeply as you like.

As a necessary consequence, in this book you'll learn how to read and write
proofs. The simplest and broadest truth about mathematics is that it revolves
around proofs. Proofs are both the primary vehicle of insight and the
fundamental measure of judgment. They are the law, the currency, and the fine
art of mathematics. Most of what makes mathematics mysterious and opaque---the
rigorous definitions, the notation, the overloading of terminology, the
mountains of theory, and the unspoken obligations on the reader---is due to the
centrality of proofs. A dominant obstacle to learning math is an unfamiliarity
with this culture.  In this book I'll show you why proofs are so important,
cover the basic methods, and display examples of proofs in each chapter. To be
sure, you don't have to understand every proof to finish this book, and you
will probably be confounded by a few. Embrace your humility. I hope to convince
you that each proof contains layers of insight that are genuinely worthwhile,
and that no single person can see the complete picture in a single sitting.  As
you grow into mathematics, the act of reading even previously understood proofs
provides both renewed and increased wisdom. So long as you identify the value
gained by your struggle, your time is well spent.

I'll also teach you how to read between the mathematical lines of a text, and
understand the implicit directions and cultural cues that litter textbooks and
papers. As we proceed through the chapters, we'll gradually become more terse,
and you'll have many opportunities to practice parsing, interpreting, and
understanding math. All of the topics in this book are explained by hundreds of
other sources, and each chapter's exercises include explorations of concepts
beyond these pages. In addition, I'll discuss how mathematicians approach
problems, and how their process influences the culture of math.

You will not learn everything you want to know in this book, nor will you learn
everything this book has to offer in one sitting. Those already familiar with
math may find early chapters offensively slow and detailed. Those genuinely new
to math may find the later chapters offensively fast. This is by design. I want
you to be exposed to as much mathematics as possible, to learn the definitions
of central mathematical ideas, to be introduced to notations, conventions, and
attitudes, and to have ample opportunity to explore topics that pique your
interest.

A number of topics are conspicuously missing from this book, my negligence of
which approaches criminal. Except for a few informal cameos, we ignore complex
numbers, probability and statistics, differential equations, and formal logic.
In my humble opinion, none of these topics is as fundamental for mathematical
computer science as those I've chosen to cover. After becoming comfortable with
the topics in this book, for example, probability will be very accessible. The
chapter on eigenvalues will include a miniature introduction to differential
equations. The chapter on groups will briefly summarize complex numbers.
Probability will echo in your brain when we discuss random graphs and machine
learning. Moreover, many topics in this book are prerequisites for these other
areas.  And, of course, as a single human self-publishing this book on nights
and weekends, I have only so much time.

The first step on our journey is to confirm that mathematics has a culture
worth becoming acquainted with. We'll do this with a comparative tour of the
culture of software that we understand so well.

\mainmatter{}
